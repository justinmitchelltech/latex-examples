\documentclass{article}

% General Setup
\usepackage[margin=0.5in]{geometry}

% for Images
\usepackage{graphicx}
\graphicspath{ {./images/} }

% for Links
\usepackage{hyperref}
\hypersetup{
    colorlinks=true,
    linkcolor=blue,
    filecolor=magenta,      
    urlcolor=cyan,
    }
    
% for Code
\usepackage{listings}
\usepackage{xcolor}

\definecolor{codegreen}{rgb}{0,0.6,0}
\definecolor{codegray}{rgb}{0.5,0.5,0.5}
\definecolor{codepurple}{rgb}{0.58,0,0.82}
\definecolor{backcolour}{rgb}{1,1,1}

\lstdefinestyle{mystyle}{
    backgroundcolor=\color{backcolour},   
    commentstyle=\color{codegreen},
    keywordstyle=\color{blue},
    numberstyle=\tiny\color{codegray},
    stringstyle=\color{codepurple},
    basicstyle=\ttfamily\footnotesize,
    breakatwhitespace=false,         
    breaklines=true,                 
    captionpos=b,                    
    keepspaces=true,                 
    numbers=left,                    
    numbersep=5pt,                  
    showspaces=false,                
    showstringspaces=false,
    showtabs=false,                  
    tabsize=2
}

\lstset{style=mystyle}


%--------------------------------------------------------------------------------------------------
%	Beginning of Document
%--------------------------------------------------------------------------------------------------
\begin{document}


%------------------------------------------------------------------------------
\section{Basic Report Writing}
\subsection{Subsection}
\subsubsection{Subsubsection}
With some text in it.
\subsubsection{Another subsubsection}
\paragraph{}
Hello world. Here's some text in the first paragraph of this subsubsection.
\paragraph{}
Here's another paragraph in the same subsubsection. 


%------------------------------------------------------------------------------
\section{Math}
\subsection{Some Random Equation}
$y = \sqrt[3]{\frac{5x}{x^2}}$

\paragraph{}
This is an inline math equation: $y = x^2 + 2x + 4$ ...in the paragraph following Some Random Equation.


%------------------------------------------------------------------------------
\section{Figures and Images}
\begin{figure}[h]
\caption{A funny LaTeX meme}
\centering
\includegraphics[width=0.5\textwidth]{latex-meme.jpg}
\end{figure}

For more examples and settings, \href{http://overleaf.com/learn/latex/Inserting_Images}{click this link.}

\pagebreak


%------------------------------------------------------------------------------
\section{Lists}

\subsection{Non-numbered List}
\begin{itemize}
	\item First item
	\item Second item
	\begin{itemize}
		\item First second item
		\item Second second item
	\item[!] Last item with custom exclamation point marker	
	\end{itemize}
\end{itemize}

\subsection{Numbered Lists}
\begin{enumerate}
	\item Numbered
	\item list of 
	\item things
	\begin{enumerate}
		\item first thing
		\item second thing
	\end{enumerate}
	\item[!] exclaimed item
	\item[NOTE] this item
	\item[$\rightarrow$] custom arrow bullet in list
	\item list cont.
\end{enumerate}


%------------------------------------------------------------------------------
\section{Tables}
\begin{center}
\begin{tabular}{ ||c|c|c|| } 
 \hline
 \textbf{head1} & \textbf{head2} & \textbf{head3} \\ 
 \hline
 \hline
 cell4 & cell5 & cell6 \\ 
 \hline
 cell7 W/kg  & 
 cell8 N  & 
 cell9 seconds \\ 
 \hline
\end{tabular}
\end{center}


%------------------------------------------------------------------------------
\section{Code}

\lstinputlisting[language=Python]{code/script_prepare-data.py}

\vspace{5mm}
For more code examples and parameters, \href{https://www.overleaf.com/learn/latex/Code_listing}{click this link.}


\end{document}
%--------------------------------------------------------------------------------------------------
%	End of Document
%--------------------------------------------------------------------------------------------------
